\documentclass[lettersize,journal]{IEEEtran}
\usepackage{amsmath,amsfonts}
\usepackage{algorithmic}
\usepackage{algorithm}
\usepackage{array}
\usepackage[caption=false,font=normalsize,labelfont=sf,textfont=sf]{subfig}
\usepackage{textcomp}
\usepackage{stfloats}
\usepackage{url}
\usepackage{verbatim}
\usepackage{graphicx}
\usepackage{cite}
\hyphenation{op-tical net-works semi-conduc-tor IEEE-Xplore}
% updated with editorial comments 8/9/2021

%%%%%%%%%%%%%%%%%%%%%%%%%%%%%%%%%%%%%%%%%%%%%%%%%%%%%%%
% URL line breaks in references
\def\UrlBreaks{\do\/\do-}

% todos
\usepackage{xcolor}
\newcommand{\todo}[1]{\textcolor{red}{TODO: #1}\PackageWarning{TODO:}{#1!}}
%%%%%%%%%%%%%%%%%%%%%%%%%%%%%%%%%%%%%%%%%%%%%%%%%%%%%%%

\begin{document}
	
	\title{Monitoring of Scientific Workflows}
	
	\author{Julian Legler, Marcin Ozimirski, Matthias Seiler
		% <-this % stops a space
		\thanks{This paper was produced by students at TU Berlin in the course ''Master Project: Distributed Systems''}% <-this % stops a space
		%\thanks{Manuscript received April 19, 2021; revised August 16, 2021.}
	}
	
	% The paper headers
	\markboth{PJ DS - Monitoring of Scientific Workflows}%
	{Shell \MakeLowercase{\textit{et al.}}: A Sample Article Using IEEEtran.cls for IEEE Journals}
	
	%\IEEEpubid{0000--0000/00\$00.00~\copyright~2021 IEEE}
	
	\maketitle
	
	\begin{abstract}
		
		The emergence of Scientific Workflow Management Systems (SWMS) allowed for speeding up scientific progress in different fields\cite{challengesSW}. The main task of such systems is to represent and manage complex distributed scientific computations. They are designed to handle datasets and examine them in a programmatical way. The software can process given jobs effectively and provide results in a requested form. However, nowadays' scientific calculations can involve hundreds of stages, each integrating several models and data sources created by various groups. Because of their complexity, they require a significant amount of resources to be processed. Such projects are well suited to a cloud environment since it allows for scalability in the event of increased resource demand. In these situations, system monitoring software is essential because it provides a better understanding of how resources are utilized when consecutive tasks are being executed. It can help to decrease the costs by choosing the best suitable provider or adjusting parts of the code for a faster execution time. Unfortunately, the built-in monitoring features offered by the SWMS developers and cloud providers deliver only high-level metrics, such as CPU per task, which is not always enough for a detailed resource usage analysis. Therefore the implementation of software allowing for monitoring on a system process level stays on the shoulders of the end users. 
		
	\end{abstract}
	
	\begin{IEEEkeywords}
		Scientific Workflow Management Systems, Monitoring, eBPF
	\end{IEEEkeywords}
	
	\section{Introduction}
	A scientific workflow management system (SWMS) is a commonly used software solution in the field of computational science. It can execute a series of calculations to examine (big) data sets in a structured and organized manner. Such computations are commonly called workflows consisting of tasks related to one another, forming a directed acyclic graph. Management System is responsible for a transparent orchestration and execution of specified tasks, considering dependencies between them. The software allows for an efficient way to process and extract information from the examined data sets, leading to scientific advances in various fields such as biology, physics, and astronomy\cite{challengesSW}. Yet, such workflows are often resource-intensive and require a distributed, dynamic, and scalable architecture to achieve meaningful results in a reasonable amount of time. The cloud and containerization paradigms offer several advantages when deploying these systems, such as accessibility, flexibility, and scalability, which allow workflows to be deployed and executed dynamically. It is a crucial feature as it is expected from a workflow to change its resource needs throughout the execution, depending on an input data set, the parallelism of computed tasks, or their level of complexity. The available resource pool can scale to adjust the computing resources throughout the workflow's execution process. However, depending on the technological solutions offered by cloud services, hardware and software factors can vary significantly from one provider to another, which leads to different processing speeds, calculation accuracy, or resource utilization\cite{compareCloud}. Built-in monitoring solutions operate at a high level of abstraction. They provide information on the amount of time and resources needed for the computation of a particular task or resource utilization on the active virtual machines at any given time. Unfortunately, for more intense calculations, like in the machine learning field, where multiple system processes can run parallelly in a distributed manner for a relatively long time. Data delivered by the built-in tools are often insufficient to provide enough information for deep analysis and detection of bottlenecks or finding opportunities for code optimization. In such scenarios, the more detailed (low-level) metrics available, the better conclusions can be drawn regarding the performance of the current solution, for example, allowing one to choose hardware best suited for the currently performed analyses. Possessing exact information on the number of I/O operations, CPU load, or memory usage per system process allows for a more accurate analysis of executed workflows (or their tasks) together with the infrastructure on which the calculations are performed. It can open doors to better workflow optimization and overall performance of the system and the code. However, the implementation of solutions capable of gathering such information or extension of the already existing ones rests on the shoulders of the end user. This project intends to provide a solution allowing the collection of high- and low-level metrics, independently of underlying cloud technology and an SWMS used for computational science. The following report is structured as follows
	
	
	
	\section{Related Work}
	Lorem \todo{stuff} ipsum
	
	\section{Implementation}
	jkjlkjlkjlkjlkjlkjlk
	
	\section{Conclusion}
	\includegraphics[width=\linewidth]{images/placeholder.jpg}
	\todo{remove placehodler}
	
	\section*{Acknowledgments}
	We thank our supervisor Sören Becker for his enourmous support and patience. 
	
	% References
	\bibliography{bib}
	\bibliographystyle{IEEEtran}
	
\end{document}

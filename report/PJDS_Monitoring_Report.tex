\documentclass[lettersize,journal]{IEEEtran}
\usepackage{amsmath,amsfonts}
\usepackage{algorithmic}
\usepackage{algorithm}
\usepackage{array}
\usepackage[caption=false,font=normalsize,labelfont=sf,textfont=sf]{subfig}
\usepackage{textcomp}
\usepackage{stfloats}
\usepackage{url}
\usepackage{verbatim}
\usepackage{graphicx}
\usepackage{cite}
\hyphenation{op-tical net-works semi-conduc-tor IEEE-Xplore}
% updated with editorial comments 8/9/2021

%%%%%%%%%%%%%%%%%%%%%%%%%%%%%%%%%%%%%%%%%%%%%%%%%%%%%%%
% URL line breaks in references
\def\UrlBreaks{\do\/\do-}

% todos
\usepackage{xcolor}
\newcommand{\todo}[1]{\textcolor{red}{TODO: #1}\PackageWarning{TODO:}{#1!}}
%%%%%%%%%%%%%%%%%%%%%%%%%%%%%%%%%%%%%%%%%%%%%%%%%%%%%%%

\begin{document}
	
	\title{Monitoring of Scientific Workflows}
	
	\author{Julian Legler, Marcin Ozimirski, Matthias Seiler
		% <-this % stops a space
		\thanks{This paper was produced by students at TU Berlin in the course ''Master Project: Distributed Systems''}% <-this % stops a space
		%\thanks{Manuscript received April 19, 2021; revised August 16, 2021.}
	}
	
	% The paper headers
	\markboth{PJ DS - Monitoring of Scientific Workflows}%
	{Shell \MakeLowercase{\textit{et al.}}: A Sample Article Using IEEEtran.cls for IEEE Journals}
	
	%\IEEEpubid{0000--0000/00\$00.00~\copyright~2021 IEEE}
	
	\maketitle
	
	\begin{abstract}
		
	\end{abstract}
	
	\begin{IEEEkeywords}
		Scientific Workflow Management Systems, Monitoring, eBPF
	\end{IEEEkeywords}
	
	\section{Introduction}
	A scientific workflow management system (SWMS) is a commonly used software solution in the field of computational science. It can execute a series of calculations to examine (big) data sets in a structured and organized manner. Such computations are commonly called workflows consisting of tasks related to one another, forming a directed acyclic graph. Management System is responsible for a transparent orchestration and execution of specified tasks, considering dependencies between them. The software allows for an efficient way to process and extract information from the examined data sets, leading to scientific advances in various fields such as biology, physics, and astronomy \cite{challengesSW}. Yet, such workflows are often resource-intensive and require a distributed, dynamic, and scalable architecture to achieve meaningful results in a reasonable amount of time. The cloud computing paradigm offers several advantages when deploying these systems, such as accessibility, flexibility, and scalability, which allow workflows to be deployed and executed dynamically. It is a crucial feature as it is expected from a workflow to change its resource needs throughout the execution, depending on an input data set, the parallelism of computed tasks, or their level of complexity. The available resource pool can scale to adjust the computing resources throughout the workflow's execution process. However, depending on the technological solutions offered by cloud services, hardware and software factors can vary significantly from one provider to another, which can lead to different processing speeds, calculation accuracy, or resource utilization [reference https://ieeexplore.ieee.org/abstract/document/8355463]. Built-in monitoring solutions operate at a high level of abstraction. They provide information on the amount of time and resources needed for a calculation of a particular task or resource usage on the active virtual machines at a given time point, yet, for more complex jobs consisting of multiple system processes running in a distributed manner, such information is not very reliable. Some scenarios need more low-level metrics, taking into account the consumption of resources per system process and not the SWMS task. Monitoring such properties, to choosing the technology best suited for the currently performed calculations rests on the shoulders of the end user. Possessing exact information on the number of I/O operations, CPU load, or memory usage, per system process allows for a more accurate analysis of executed workflows (or their tasks) together with the infrastructure on which the calculations are performed. This approach opens door to a better workflow and cluster optimization, and overall performance of the system. The goal of this project is to provide a solution allowing the collection of low-level metrics, independently of an SWMS used.




[1] Y. Gil, E. Deelman, M. Ellisman, T. Fahringer, G. Fox, D. Gannon, C. Goble, M. Livny, L. Moreau, J. Myers, Examining the challenges of scientific workflows, IEEE Comput. 40 (12) (2007) 26–34.
	
	\section{Related Work}
	Lorem \todo{stuff} ipsum
	
	\section{Conclusion}
	\includegraphics[width=\linewidth]{images/placeholder.jpg}
	\todo{remove placehodler}
	
	\section*{Acknowledgments}
	We thank our supervisor Sören Becker for his enourmous support and patience. 
	
	% References
	\bibliography{bib}
	\bibliographystyle{IEEEtran}
	
\end{document}


